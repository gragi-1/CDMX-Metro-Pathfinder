%Introducción, con una descripción del problema a resolver y objetivos a alcanzar
\section{Introducción}
En esta memoria se verán los aspectos más importantes de la motivación, aspectos espefícios y desarrollo del proyecto, además de código descargable y ejectuable del trabajo, y por último, unas conclusiones sobre como nos ha ayudado personalmente a todos los integrantes este trabajo.
\subsection{Problema original} 
El proyecto a desarrollar consistía en una aplicación de transporte para los usuarios de la Ciudad de México, que les permitiese ver, entre otras cosas, los trayectos óptimos entre cualesquiera estaciones que escojan de entre las líneas 1,3,7,9 y 12 de la red Metro de CDMX.
\subsection{Objetivos}
Nuestros objetivos con el desarrollo de esta aplicación no fue solo la de dar una manera de encontrar la ruta más corta entre estaciones, si no también dar otra información esencial al consumidor, como el tiempo estimado en completar la ruta, además de la cantidad de transbordos a realizar y una opción donde puedas ver esa ruta estación por estación, avisándote de en que estaciones se realizan esos transbordos. Por último, quisimos añadir una opción en la aplicación que permitiese al usuario buscar rutas alternas sin escaleras, sobretodo por y para las personas con problemas para la movilidad.