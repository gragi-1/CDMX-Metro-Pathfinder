%3. Descripción del desarrollo de la práctica, con indicación expresa y detallada de los cálculos efectuados para las funciones g(n) y h(n) involucradas en el algoritmo A*, descripción del diseño e implementación de la aplicación y capturas de pantalla comentadas.
\section{Desarrollo}
Para la resolución del problema principal, determinar cuál es ese camino mínimo entre estaciones dadas, se uso un enfoque desde teoría de grafos junto con el algoritmo A*, este es un método de búsqueda informada que permite encontrar el camino óptimo entre nodo inicial y nodo objetivo, por ello la representación de la red Metro como grafo.
La fórmula principal que caracteriza A* es f(n)=g(n)+h(n) y a continuación vemos como se obtuvieron ambas, g(n) y la heurística h(n).
\subsection{Obtención de g(n)}
\subsection{Obtención de h(n)}
\subsection{Estructura del Trabajo}
Una vez tenemos las funciones, solo quedó la implementación en Python de este algoritmo junto con los datos de la propia red de Metro de CDMX, además se añadió un extra de estilos que permitió crear una página web con las funciones deseadas y planteadas en (1.2) y no solo un código ejecutable.

 Tanto el script.js del programa como el estilo style.css y el index.html que se apoya en los códigos anteriores se pueden encontrar en su totalidad en el apartado siguiente.

